\documentclass{article}

\usepackage[backend=biber, style=apa]{biblatex}
\usepackage{amsmath}
\usepackage{graphicx}
\usepackage{subcaption}
\usepackage{hyperref}
\usepackage{cleveref}
\usepackage{booktabs}
\usepackage{multirow}
\usepackage{float}
\usepackage{tikz}
\usepackage{pgfplots}
\usepackage{pgfplotstable}
\usepackage{appendix}
\usepackage{smartdiagram}
\usepackage{amsfonts} 

\addbibresource{../biblio.bib}

\title{Towards an Artificial Intelligence Audit Framework for Technical Robustness, Transparency, and Fairness}

\author{Chiara Galimberti\thanks{TBD} \and Marco Repetto\thanks{CertX, Fribourg, Switzerland.}}


\begin{document}
\maketitle
\begin{abstract}
	The European Artificial Intelligence Act proposes a regulatory framework for Artificial Intelligence Systems aiming at ensuring that such a systems placed on the EU market and used in the Union are safe and respect existing law on fundamental rights and Union values.
	Although proposing a compelling legal framework such an act falls short in terms of delivering technical requirements that can be implemented by the Decision Maker.
	Moreover, the Act does not adequately address the trade-offs that an Artificial Intelligence system may have in different aspects ranging from accuracy to transparency, robustness to fairness ans so forth.
	In this paper we propose an audit framework addressing these tensions. 
	The proposed approach collects technical requirements from state-of-the-art research combines them using the Analytical Hierarchical Approach and construct a Multicriteria Optimization problem allowing the decision maker to mitigate AI risks by optimally allocating his resources. 
\end{abstract}

\section{Introduction}

Artificial Intelligence (AI) systems are becoming increasingly prevalent in our society, with applications ranging from autonomous vehicles to facial recognition. 
As AI technology continues to advance, it is crucial to ensure that these systems are safe, transparent, and fair.
In an effort to tame this powerful yet uncontrolled area, many countries started developing regulation frameworks.

Among the multitude of such frameworks the one proposed by the European Union (EU) stands out in terms of novelty and maturity.

% L'approccio risk base, magari qui metto uno schemino
One of the relevant feature of the Act is the risk-based approach to AI regulation, by classifying AI systems into four categories according to their potential impact on human rights, safety, and fundamental values:

\begin{itemize}	
    \item Prohibited AI systems are those that are considered unacceptable and contrary to the EU’s values and principles, such as AI systems that manipulate human behavior, opinions, or decisions; AI systems that exploit vulnerabilities of specific groups; AI systems that allow social scoring by governments; etc. These AI systems are banned in the EU.
    
    \item High-risk AI systems are those that are likely to cause significant harm or adverse effects on human rights, safety, or fundamental values, such as AI systems used for biometric identification; AI systems used for critical infrastructure; AI systems used for education or vocational training; AI systems used for employment or workers management; etc. These AI systems are subject to strict obligations and requirements, such as adequate risk assessment and management; high quality and traceability of data and algorithms; transparency and provision of information to users; human oversight and intervention; accuracy, robustness, and security; etc. These AI systems also have to undergo a conformity assessment by a notified body before they can be placed on the market or put into service.
    
    \item Limited-risk AI systems are those that pose some risks to human rights, safety, or fundamental values, but to a lesser extent than high-risk AI systems, such as AI systems used for chatbots; AI systems used for video games; AI systems used for image or video manipulation; etc. These AI systems are subject to transparency obligations, such as informing users that they are interacting with an AI system; disclosing the use of automated image or video manipulation; etc.
    
    \item Minimal-risk AI systems are those that pose no or negligible risks to human rights, safety, or fundamental values, such as AI systems used for spam filters; AI systems used for smart home devices; AI systems used for personal assistants; etc. These AI systems are not subject to any specific obligations or requirements under the regulation, but they have to comply with the general principles and existing laws of the EU.
\end{itemize}

Figure ... 

Given the broad scope of the legislation and the fact that the field is still evolving, a clear example been the inclusion of foundation models after the first draft, the Act does not contain any technical requirements nor means of compliance.

Such a standardization task has been given to the CEN-CENELEC Joint Technical Committee 21.

However as noted by \cite{europeancommission.jointresearchcentre._2023a} the current international standards already provide adequate coverage of the Act.

In particular dimensions such as...

Nonetheless such technical requirements can be difficult to implement by the Decision Maker which is faced by a multitude of conflicting objectives and a budget constraint.

Many authors have proposed qualitative frameworks such as Floridi and ...
\section{Related works}
Our work can be placed in the recent strand of literature developing frameworks for assessing the risk of AI systems.
Such literature is relatively new and is still in its infancy and started following the work of the High-Level Expert Group on Artificial Intelligence (AI HLEG) \cite{ec2020}.

\cite{benjamins_2021} is one of the first works that proposes a framework for the risk assessment of AI systems.
In the paper a great emphasizes is posed on the need for organizations to make deliberate choices regarding AI and its application to specific problems. What is proposed is a "choices" framework that helps organizations understand and act on the negative consequences of AI before they occur. The framework includes technical choices such as bias, explainability, agency, errors, and continuous learning, as well as generic digital technical choices like privacy, security, and safety. The paper also highlights the significance of transparency and explainability in AI systems.

In \cite{falco_2021} the emphasis is posed on the crucial role of governance and assurance in the application of highly automated systems, particularly in fields that are deemed as critical infrastructure sectors. They propose "AAA" governance principles, namely: prospective Assessments before highly automated systems are implemented; Audit trail to analyze failures and help assess accountability; and system Adherence to jurisdictional requirements for these systems. The authors underline the mandate for transparency and responsibility in handling data related to automated systems, suggesting that data should be shared publicly in an anonymous and secure way to improve system design and foster public trust. They advocate for an independent audit for such systems and propose incentives for organisations to encourage the adoption of these audits.

A different perspective is taken by \cite{vakkuri_2021}. In their paper they examine the application of AI ethics guidelines in practice and discuss the development and iterations of a method called ECCOLA for addressing AI ethics in hands-on applications. Such a framework based on the Essence language of software engineering takes the form of a card-deck format categorised by AI ethics themes. The paper also discusses practical feedback-based improvements made to the ECCOLA cards in terms of layout, readability, and visual appearance.

\cite{winter_2021} instead propose a more systematic approach.
Such an approach is centered around the conception of a comprehensive certification catalogue by TÜV AUSTRIA in collaboration with the Johannes Kepler University Linz and the Software Competence Center Hagenberg.
The foundation of this approach rests on several principles. Firstly, there is the definition of the technical distribution of the application domain, a key element that determines how an AI system should operate within its intended environment. This is followed by a risk-based definition of performance requirements, to ensure the system can meet certain minimum standards even under adverse or unusual conditions. Lastly, there is a statistically valid testing of the final model using independent random samples, providing empirical assurance that the AI system will perform reliably in real-world applications.

The first holistic and flexible framework for the risk assessment of AI systems is proposed in \cite{zicari_2021}.
The paper presents a robust and flexible framework for risk assessment of AI systems known as Z-Inspection. With a focus on applied ethics, it contrasts other toolkits which are predominantly checklists or designed for the early process of design and deployment. Z-Inspection is unique because it can be used for auditing, performing an ethical evaluation over time of already deployed AI systems, and can also be applied by investigators external to the deploying organisation.
The Z-Inspection process involves tasks such as classifying the AI system by domain and usage, reviewing domain-specific frameworks, regulations and laws, developing an evidence base, and listing potential ethical, technical, and legal issues.

\cite{floridi_2022} propose a framework for the ethical assessment of AI systems called capAI. 
CapAI stems from the dire need for ensuring that Artificial Intelligence (AI) systems operate ethically, with the responsibility placed on the organisations that develop and deploy them. It is intended to serve as an additional tool in the management toolbox of organisations to ensure their AI systems strictly adhere to specific ethical principles.
CapAI aims to encourage good software development practices and prevent the most common ethical failures as identified through research. 
Moreover, it supports good governance by following standard methodologies like the Independent Review Panel (IRP), which allows organisations to have a competitive advantage. 
It achieves this by preventing common failures, validating public claims about ethical AI processes, and protecting the organisation’s reputation.
Additionally, capAI is designed to offer practical guidance by converting high-level ethical principles into verifiable criteria to aid in the ethical construction, development, implementation, and utilisation of AI. 
This makes it effective in identifying and rectifying unethical behaviours of AI systems.
Furthermore, capAI gives providers of both ‘high-risk’ and ‘low-risk’ AI systems procedural guidance on verifying claims made about the AI systems they design and deploy. For providers of ‘high-risk’ AI systems, capAI can be used to show compliance with the EU’s Artificial Intelligence Act (AIA). 
For providers of ‘low-risk’ AI systems, they can use capAI to operationalise their commitments to voluntary codes of conduct.
In conclusion, capAI proffers an auditable and standardised process for developing, deploying and operating AI. 
By adopting capAI, the most common failure modes could be prevented, thereby improving the trustworthiness of AI systems.
\section{Methodology}
This paper aims to propose a sound AI audit framework that can be used by the DM to perform sampling and testing of a complex AI system.
The framework incorporates two relevant dimensions of the AI audit process: the tradeoffs that the AI system has to face and the scarcity of resources of the DM.

Such a result is achieved by combining the AHP to decouple the different tradeoffs and the MO to solve the sampling problem. 
In particular, polynomial GP à-là \cite{tayi_1985} is used to combine the different and conflicting objectives.

The result is a five-step process consisting of the following steps:
\begin{itemize}
    \item \textbf{System identification:} The DM defines the AI system to be audited and the context in which it will be used.
    \item \textbf{Tradeoffs prioritization:} Employing AHP the DM defines the relative importance of the tradeoffs that the AI system has to face.
    \item \textbf{Optimal allocation:} Utilising GP the DM is able to achieve an optimal sampling and is capable of allocating his resources to counter emerging risks of the AI system.
    \item \textbf{Audit:} The DM mandates the audit of the AI system according to the sampling schema obtained.
    \item \textbf{Continuous improvement:} The DM uses the results of the audit to improve the AI system.
\end{itemize}

The proposed framework is depicted in Figure \ref{fig:framework}.

\begin{figure}
    \centering
    \smartdiagram[circular diagram:clockwise]{System identification, Tradeoffs prioritization, Optimal allocation, Audit, Continuous improvement}
    \caption{The proposed framework}
    \label{fig:framework}
\end{figure}

%% STEP 1: System identification
The first step of the proposed framework is the identification of the AI system to be audited and the context in which it will be used.
This step is crucial and is a common point in many other AI audit frameworks such as CapAI and Z-Inspection.
In CapAI the identification of the AI system is a risk-based one that makes their approach more in line with the EU AI Act.
In Z-Inspection the identification is part of the assessment process and is more comprehensive than the one proposed in CapAI.

In our methodology, the identification of the AI system is pivotal to identifying two aspects of the AI system.
The first aspect is the legal basis of the AI system, in this vein we follow the risk identification approach proposed by CapAI.
Provided that the AI system is legal and bound to certain requirements, such for example transparency for chatbots in the EU under the AI Act, the DM can proceed to the next step.
The next step is the identification of the tradeoffs that the AI system has to face.
This set of tradeoffs necessarily contains the legal requirements of the AI system but it is not limited to them.
In fact, the DM can decide to include other tradeoffs that are relevant to the AI system and its particular use case.

AI tradeoffs are the tensions that an AI system has to face to achieve its goals.
In case the case of a high-risk AI system affected by the EU AI Act the necessary tradeoffs entail the following:

\begin{itemize}
    \item Accountability
    \item Human agency and Oversight
    \item Technical robustness and Safety
    \item Privacy and Data Governance 
    \item Transparency
    \item Diversity, Non-discrimination and Fairness
    \item Societal and Environmental well-being
\end{itemize}

%% STEP 2: Tradeoffs prioritization
After the identification of the AI system to be audited and the context in which it will be used the DM has to define a prioritization of the tradeoffs that the AI system has to face.
Following the best practice delineated by \cite{vetter_2023} we propose to use a binary list of tradeoffs.
However, contrary to \cite{vetter_2023} we propose to use the AHP to define the relative importance of the tradeoffs.

% Mini explanation of AHP
AHP is a decision-making tool that allows the DM to define the relative importance of a set of alternatives.
Originally proposed by \cite{saaty_1988} is a methodical, mathematical decision-making technique used for multi-criteria decision analysis.
It allows the DM to model a complex problem in a hierarchical structure showing the relationships between the decision goal, criteria, sub-criteria (if exist), and alternatives.

Firstly, a pairwise comparison matrix (PCM) is constructed for each criterion and alternative.
This begins by comparing each criterion or alternative against each other relatively based on a scale of importance (from equally important to extremely more important). 
If $a_i$ and $a_j$ are two alternatives or criteria, and $a_i$ is considered to be $n$ times as important as $a_j$, the value of $ij^{th}$ element in the matrix is $n$, and $ji^{th}$ element is $1/n$.

The PCM is denoted as $A=[a_{ij}]$ with dimensions $n \times n$, where $n$ is the number of alternatives or criteria. 

\begin{equation}
    A = 
    \begin{bmatrix}
    a_{11} & a_{12} & ... & a_{1n} \\
    a_{21} & a_{22} & ... & a_{2n} \\
    ... & ... & ... & ... \\
    a_{n1} & a_{n2} & ... & a_{nn}
    \end{bmatrix}        
\end{equation}

where, $a_{ij}$ is the importance of $i^{th}$ element over $j^{th}$ element.

This is followed by calculating a priority vector $\omega$. 
We calculate the geometric mean of each row and normalize it, resulting in the eigenvector.

\begin{equation}
\omega_i = \frac{\left(\prod_{j=1}^{n}a_{ij}\right)^{1/n}}{\sum_{k=1}^{n}\left(\prod_{j=1}^{n}a_{kj}\right)^{1/n}}
\end{equation}

where, $\omega_i$ is the priority of $i^{th}$ element.

The consistency of judgments is then calculated using a Consistency Index (CI) and a randomly generated Consistency Ratio (CR). If CR>0.1, judgments are considered inconsistent, and the process is reiterated.

\begin{equation}
CI=\frac{\lambda_{max} - n}{n - 1} \quad \text{and} \quad CR = \frac{CI}{RI}
\end{equation}

where, $\lambda_{max}$ is the maximum eigenvalue of $A$, and $CI$ and $RI$ are Consistency Index and Random Index.

Finally, the criteria and alternative weights are combined to calculate the final overall scores for the alternatives.

%% STEP 3: Optimal allocation
The previous step allows the DM to define the relative importance of the tradeoffs impacting a given AI system.
The next step is to allocate his resources to mitigate the risks of the AI system.

In practice, such a task is performed by either internal or external auditors whose aim is to test the system and provide a report to the DM.
For highly complex systems such a task is challenging and it is not always possible to audit the system in its entirety.
Such notion of scarcity of resources is well known in the financial audit literature but rather unexplored in the AI audit literature.
In fact, neither \cite{floridi_2022} nor \cite{zicari_2021} mention the notion of scarcity of resources.
In our framework, instead, we assume that the DM has a limited amount of resources and that he has to decide how to allocate them to maximize the quality of the audit.

Audit quality, no matter the target, is a multidimensional concept involving a series of objectives. 
In \cite{kinney_1972} they envision several objectives, ranging from correctness to preventiveness.
Inspired by \cite{tayi_1985} we propose to frame the problem as a MO problem and to solve it through GP.

% Mini intro to goal programming
GP is a multi-objective optimization technique that aims to minimize deviations from a set of predefined goals, which are typically expressed as mathematical constraints. 
GP is a popular technique in decision sciences, as it allows decision-makers to explicitly consider multiple and often conflicting objectives in a single optimization model.

A general GP model can be formulated as follows:

\begin{align}
\text{minimize} & \sum_{i=1}^m w_i d_i^+ + d_i^- \\
\text{subject to} & \sum_{j=1}^n a_{ij} x_j - d_i^+ + d_i^- = g_i, &i=1,\dots,m \\
& x_j \ge 0, &j=1,\dots,n
\end{align}

where $m$ is the number of goals, $n$ is the number of decision variables, $w_i$ is the weight of the i-th goal, $d_i^+$, and $d_i^-$ are the positive and negative deviations from the i-th goal. $a_ij$ is the coefficient of the j-th decision variable in the i-th goal constraint, $x_j$ is the j-th decision variable, and $g_i$ is the target value of the i-th goal.

Our setting reframes the problem of sampling as discussed by \cite{tayi_1985} with the caveat that instead of error rates, we utilize the weights derived from the AHP.

% Problem formulation
In our problem formulation, we have a battery of tests and documents that the AI system possesses. 
These tests and documents cover the different dimensions that compose what can be defined a trustworthy AI system.

These $N$ tests and documents can be indexed by $A$ and $B$ respectively.
Where $A$ identifies the dimensions coming from the AHP and $B$ identifies the specific risk magnitude of a given aspect the test or document is covering.

In this setting $n_{ij}$ is the number of tests or documents that the AI system possesses for a given dimension $i \in A$ and magnitude $j \in B$ and $k \in A \times B$ identifies a unique tuple.
The decision variable instead is $x_{ij}$ or $x_{k}$.
From these definitions is worth observing that $x_{ij} \leq n_{ij}$ and when $x_{ij} = n_{ij}$ the meaning is that we are mandating a complete exploration of that particular dimension/magnitude tuple.
Moreover we define $w{k} = w_{ij} = \frac{n_ij}{N}$ as the proportion of tests/documents for a given dimension/magnitude tuple with the marginal proportion for the dimension $i$ being $w_i = \sum_{j=1}^{|B|} w_{ij}$.

Given this setting the first trivial objective is the budget constraint:

\begin{equation}
    \sum_{k \in A \times B} c_{k} x_{k} - d_1^+ \leq g_1
\end{equation}

where $g_1$ is the budget constraint and $c_{k}$ is the cost of a associated with the assessment of a particular test/document.

The next two objectives composes the sample representativenes such that we assure good coverage of all the relevant dimensions as well as magnitudes.

\begin{equation}
\sum_{i \in A} \frac{w^2_i \omega_i (1-\omega_i)}{\sum_{j \in B} x_{ij}} - d_2^+ \leq g_2
\end{equation}

\begin{equation}
\sum_{k \in A \times B} \frac{w^2_{k} \omega_k(fD_k)^2}{x_{k}} - d_3^+ \leq g_3
\end{equation}

The third objective focuses on consistency with respect to the preference scheme of the DM defined while using AHP, that is:

\begin{equation}
\sum_{k \in A \times B}  p_k(n_{k} - x_{k}) - d_4^+ \leq g_4
\end{equation}

The fourth objective is about being protective and can be decomposed into two sub objectives:

\begin{equation}
\sum_{k \in A \times B} D_k(n_{k} - x_{k}) - d_5^+ \leq g_5
\end{equation}

and 

\begin{equation}
\sum_{k \in A \times B} p_k D_k (n_{k} - x_{k}) - d_6^+ \leq g_6
\end{equation}

The remaining hard constraints being:

\begin{equation}
0 \leq x_k \leq n_k
\end{equation}

and

\begin{equation}
d_q^+ \in \mathbb{R^+} \quad \forall \quad q=1\dots6
\end{equation}

The resulting objective function collecting all the desiderata take the following polynomial form:

\begin{equation}
    Z =  \min_{x} [(d_1^+)^{p1} + (d_2^+)^{p2} + (d_3^+)^{p3} + (d_4^+)^{p4} + (d_5^+)^{p5} + (d_6^+)^{p6}]
\end{equation}

Solving this MO problem allows the DM to allocate his resources optimally and to further assess the risks of the AI system by placing more emphasis on certain aspects of the AI system.

After the optimal allocation of resources, the DM can proceed to the audit of the AI system, which is the next step of the proposed framework.
During this process, new information can be gathered and the DM can use it to improve the AI system.
Moreover, the auditing process can result in additional requirements that the AI system has to face.

This is the reason behind our framework being circular and similar to \cite{zicari_2021}.
The process must be repeated continuously to ensure that the AI system is safe and trustworthy throughout its lifecycle.
\include{sections/evaluation}
\include{sections/discussion}
\include{sections/conclusion}

\printbibliography

\begin{appendices}
	\section{European Artificial Intelligence Act: an historical perspective}
The EU has been a pioneer and leader in the field of AI regulation. 
It has been interested in developing and regulating AI for a long time, and has adopted a series of initiatives and documents that reflect its vision and strategy for AI. 
The EU’s approach to AI is based on two main pillars: trust and excellence. 
Trust means that AI systems should respect the EU’s values and fundamental rights, and comply with the relevant laws and regulations.
Excellence means that AI systems should be innovative and competitive, and contribute to social and economic progress.

The EU’s interest in AI dates back to the 1980s, when it launched its first research program on AI, called ESPRIT (European Strategic Programme for Research in Information Technology). 
The EU has also established several networks and platforms for collaboration and coordination on AI research and innovation, such as the European Research Area Network on Artificial Intelligence (ERA-Net AI), the European AI Alliance, the European Laboratory for Learning and Intelligent Systems (ELLIS), etc.

The EU’s interest in AI regulation also emerged in the late 1980s, when it adopted its first directive on data protection, which set the basic principles for the processing of personal data in the EU. 
Since then, the EU has developed a comprehensive legal framework for data protection and privacy, which is relevant for many aspects of AI, such as data collection, processing, storage, sharing, etc.
The most recent and important piece of legislation in this area is the General Data Protection Regulation (GDPR), which came into force in 2018. 

In addition to data protection and privacy, the EU has also addressed other legal issues that are relevant for AI regulation, such as liability, consumer protection, intellectual property rights, human rights, etc.

However, the EU also recognized the need to have a more specific and tailored legal framework for AI regulation, that could address the particular challenges and opportunities of AI systems, and harmonize the existing and emerging national laws and initiatives on AI across its member states. 
Based on these considerations, the EU proposed the Artificial Intelligence Act (AI Act) in April 2021, as part of its broader Digital Strategy. 

The AI Act is the first comprehensive legal proposal on AI in the world.
It aims to create an ecosystem of trust and excellence for AI in the EU, by setting harmonized rules and requirements for AI providers, users, and authorities. 
It also proposes a governance structure and enforcement mechanism for ensuring compliance and accountability. 
The AI Act is currently under discussion by the European Parliament and the Council of the EU, which will have to agree on a final text before it can become law.

\end{appendices}

\end{document}
