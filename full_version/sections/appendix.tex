\section{European Artificial Intelligence Act: an historical perspective}
The EU has been a pioneer and leader in the field of AI regulation. 
It has been interested in developing and regulating AI for a long time, and has adopted a series of initiatives and documents that reflect its vision and strategy for AI. 
The EU’s approach to AI is based on two main pillars: trust and excellence. 
Trust means that AI systems should respect the EU’s values and fundamental rights, and comply with the relevant laws and regulations.
Excellence means that AI systems should be innovative and competitive, and contribute to social and economic progress.

The EU’s interest in AI dates back to the 1980s, when it launched its first research program on AI, called ESPRIT (European Strategic Programme for Research in Information Technology). 
The EU has also established several networks and platforms for collaboration and coordination on AI research and innovation, such as the European Research Area Network on Artificial Intelligence (ERA-Net AI), the European AI Alliance, the European Laboratory for Learning and Intelligent Systems (ELLIS), etc.

The EU’s interest in AI regulation also emerged in the late 1980s, when it adopted its first directive on data protection, which set the basic principles for the processing of personal data in the EU. 
Since then, the EU has developed a comprehensive legal framework for data protection and privacy, which is relevant for many aspects of AI, such as data collection, processing, storage, sharing, etc.
The most recent and important piece of legislation in this area is the General Data Protection Regulation (GDPR), which came into force in 2018. 

In addition to data protection and privacy, the EU has also addressed other legal issues that are relevant for AI regulation, such as liability, consumer protection, intellectual property rights, human rights, etc.

However, the EU also recognized the need to have a more specific and tailored legal framework for AI regulation, that could address the particular challenges and opportunities of AI systems, and harmonize the existing and emerging national laws and initiatives on AI across its member states. 
Based on these considerations, the EU proposed the Artificial Intelligence Act (AI Act) in April 2021, as part of its broader Digital Strategy. 

The AI Act is the first comprehensive legal proposal on AI in the world.
It aims to create an ecosystem of trust and excellence for AI in the EU, by setting harmonized rules and requirements for AI providers, users, and authorities. 
It also proposes a governance structure and enforcement mechanism for ensuring compliance and accountability. 
The AI Act is currently under discussion by the European Parliament and the Council of the EU, which will have to agree on a final text before it can become law.
